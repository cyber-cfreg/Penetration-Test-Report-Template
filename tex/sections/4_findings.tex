%%%%%%%%%%%%%%%%%%%%%%%%%%%%%%%%%%%%%%%%%%%%%%%%%%%%%%%
%          Penetration Test Report Template           %
%                     cyber@cfreg                     %
%             https://hack.newhaven.edu/              %
%                                                     %
%                    Contributors:                    %
% TJ Balon - https://github.com/balon                 %
% Samuel Zurowski - https://github.com/samuelzurowski %
% Charles Barone - https://github.com/CharlesBarone   %
%%%%%%%%%%%%%%%%%%%%%%%%%%%%%%%%%%%%%%%%%%%%%%%%%%%%%%%
% This section is where you should supply any technical findings from your engagement.
% It is structured with summary tables followed by a summary for each vulnerability.
% These assessments are stored in tex/findings/ in their applicable directory depending upon the severity.
% tex/findings/findings_template.tex can be duplicated to easily make new assessments.
% It is recommended that you make the section label on line 2 of new assessments the same as the file name for organization purposes.
\section{Technical Findings}
\label{sec:tech}
This table shows the total number of vulnerabilities found during the penetration test engagement. The vulnerabilities are categorized based on the risk level. The risk levels were calculated using the Common Vulnerability Scoring System (CVSS) \cite{cvssdocs}.

\begin{center}
\textbf{Risk Level and Total Number of Discovered Vulnerabilities}
\end{center}

\setlength\arrayrulewidth{1.25pt}

% As of the current version of this template, the numbers in this table must manually be set
\begin{table}[h]
\renewcommand{\arraystretch}{2}
    \centering
    \begin{tabular}{|>{\large}p{11em}|>{\large}p{5em}|>{\large}p{5em}|>{\large}p{5em}|>{\large}p{6em}|}\hline
         Severity & \cellcolor{green}Low \newline (0.1-3.9) & \cellcolor{yellow}Moderate (4.0-6.9) & \cellcolor{orange}High (7.0-8.9) & \cellcolor{red}Critical (9.0-10.0)\\\hline
        Vulnerability Count& 1 & 1 & 1 & 1 \\\hline
    \end{tabular}
\end{table}

% This table provides a summary of Vulnerabilities by their Base Scores.
% https://nvd.nist.gov/vuln-metrics/cvss/v3-calculator
% The \vuln{}{}{}{} command allows you to automatically fill these rows with proper cell colors.
% Syntax: \vuln{vulnerability name}{base score}{impact score}{exploitability score}
% Example: \vuln{Lack of MariaDB Authentication}{7.5}{5}{10}
% Ensure these rows are in the right order, highest to lowest base score (top to bottom)
% Also ensure that the tex files in tex/findings/ are included in the same order as this table
\begin{vulntable}
    \vuln{Lack of PostgreSQL Authentication}{9}{8}{10}
    \vuln{Lack of MariaDB Authentication}{7.5}{5}{10}
    \vuln{Payment Transaction Enumeration}{4.5}{4}{5}
    \vuln{ScadaBR Reflected XSS (Username)}{1.25}{1}{1.5}
\end{vulntable}


% Here you can include your vulnerability assessments
% Ensure these are in the same order as the table above
% If one of these subsections has multiple input files, put a \newpage in between the inputs.
% tex/findings/finding_template.tex is available as a template for making new assessments.
\newpage
\subsection{Critical Risk}
%%%%%%%%%%%%%%%%%%%%%%%%%%%%%%%%%%%%%%%%%%%%%%%%%%%%%%%
%          Penetration Test Report Template           %
%                     cyber@cfreg                     %
%             https://hack.newhaven.edu/              %
%                                                     %
%                    Contributors:                    %
% TJ Balon - https://github.com/balon                 %
% Samuel Zurowski - https://github.com/samuelzurowski %
% Charles Barone - https://github.com/CharlesBarone   %
%%%%%%%%%%%%%%%%%%%%%%%%%%%%%%%%%%%%%%%%%%%%%%%%%%%%%%%
\subsubsection{Lack Of PostgreSQL Authentication}
\label{sec:creds}
\noindent

\vulntext{9.5}

\noindent


\color{black}{}
\textbf{Description}:

The host Charley on the network did not require password authentication for the postgres user in PostgreSQL. As a result, attackers can access all databases on charley and enumerate data found. The postgres user has full control over the database within the host.

\Figure[placement=h, width=\textwidth, frame, caption={User postgres does not require a password to authenticate.}]{images/findings/critical/example_critical_finding1.png}

\noindent
\textbf{Potential Business Impact}:

The data stored within this database contained unencrypted database information which is a direction violation of PCI DSS more information information can be found in Section \ref{sec:compliance}. Failures of PCI DSS can result in fines and other punishments. Each security incidents and breaches can result of a \$500,000 fine \cite{pcidssfineexample}. Figure \ref{fig:images/findings/critical/example_critical_finding2.png} shows that credit card information was stored encrypted.

\Figure[placement=h, width=\textwidth, frame, caption={PostgreSQL Billing Table stored in the clear.}]{images/findings/critical/example_critical_finding2.png}


\noindent
\textbf{Affected Host}:

    Eggdicator (10.0.17.10)
    
    Scrumdiddlyumptious (10.0.17.12)
    
    Charley (10.0.17.14)

\noindent
\textbf{Exploitation Details}:
    
    A user who can connect to 10.0.17.14 can connect to the postgresql server by running the following command:
    
\begin{lstlisting}[language=bash,frame=single,showstringspaces=false]
psql -U postgres -p 5432 -h 10.0.17.14
\end{lstlisting}


\noindent
\textbf{Recommended Remediation}:

    Harden the PostgreSQL server to require password authentication. Additionally, having firewall access controls to restrict what respective IP addresses can access the database would provide an additional layer of security.
    
\begin{lstlisting}[language=bash,frame=single,showstringspaces=false]
ALTER USER postgres PASSWORD `B3tt3rP@ssw0rd';
\end{lstlisting}

    Additionally, the PostgreSQL instance could be further hardened by making rules in the \textit{pg\_hba.conf} file to only allow for authentication from certain hosts. More information about this configuration file can be found in the references for this section.

\noindent
\textbf{References}:

    \url{https://www.postgresql.org/docs/13/auth-password.html}
    
    \url{https://www.postgresql.org/docs/9.2/auth-pg-hba-conf.html}


\newpage


\newpage
\subsection{High Risk}
%%%%%%%%%%%%%%%%%%%%%%%%%%%%%%%%%%%%%%%%%%%%%%%%%%%%%%%
%          Penetration Test Report Template           %
%                     cyber@cfreg                     %
%             https://hack.newhaven.edu/              %
%                                                     %
%                    Contributors:                    %
% TJ Balon - https://github.com/tjbalon               %
% Samuel Zurowski - https://github.com/samuelzurowski %
% Charles Barone - https://github.com/CharlesBarone   %
%%%%%%%%%%%%%%%%%%%%%%%%%%%%%%%%%%%%%%%%%%%%%%%%%%%%%%%
% Mysql no auth
\subsubsection{Lack of MariaDB Authentication}
\label{sec:mysql_no_auth}

\noindent
\vulntext{7.5}

% Description ================================================================================ 
\color{black}{}
\noindent
\textbf{Description}:

    Unauthenticated access to a MySQL database permits access/modification to sensitive datasets, including the following:
    
    \begin{itemize}
        \item Customer Accounts and passwords (Base64 encoded).
        \item Customer PII - includes phone numbers, address, and payments (including amounts).
        \item Invoices and payments.
        \item Creation of administrator accounts for \cptc's croissant marketplace.
        \item Insertion, deletion, and modification of all data within the database.
    \end{itemize}

    \Figure[placement=h, width=\textwidth, frame, caption={Passwordless Root MariaDB Access}]{images/findings/high/example_high_finding1.png}

% Impact ================================================================================ 
\noindent
\textbf{Potential Business Impact}:

    This host (Charley) can severely impact the  Confidentiality, Integrity, and Availability (CIA) of transactions within the warehouse management systems. Improper encoding schemes result in an environment in which all of the \cptc\ store website (Scrumdiddlyumptious) users' passwords can be decoded from base64.
    
    All accounts can also be modified to granted administrator level access on the \cptc\ store. As data is parsed through the root user, transactions, account details, items (for sale), and other information which is integral to \cptc's ability to sell its products online may be subject to unauthorized modification. This data is not directly accessed by the web store host (Scrumdiddlyumptious), but rather through an API endpoint host (Whatchamacallit).
    \noindent
    
    
    \textbf{Affected Hosts}:
    
    Charley (10.0.17.14)
    
    Scrumdiddlyumptious (10.0.17.12)
    
    Whatchamacallit (10.0.17.13)

% Details ================================================================================ 
\noindent
\textbf{Exploitation Details}:

    \teamname\ used the MySQL command line application to query the unauthenticated database as the root user.  All customer information could be obtained without modifying the database schema or contents.  This user also had write permissions, meaning a malicious actor could arbitrarily modify database contents.
    
    Figure \ref{fig:images/findings/high/example_high_finding2.png} shows the connection from the PostgreSQL server and initial shell commands.
    
    \Figure[placement=h, width=\textwidth, frame, caption={MariaDB tokens Table Dump}]{images/findings/high/example_high_finding2.png}

% Remediation ================================================================================ 
\noindent
\textbf{Recommended Remediation}:

A minimal number of hosts should be able to interact on the network with the MySQL database. By limiting network access to only those required hosts, potential attacks against the MySQL service are minimized.  The following iptables commands can be used to enable access controls and limit which hosts can reach MySQL:

\begin{lstlisting}[language=bash,frame=single,showstringspaces=false]
iptables -A INPUT -p tcp --dport 3306 -s <IP> -j ACCEPT
iptables -A INPUT -p tcp --dport 8000 -j DROP
\end{lstlisting}

If a firewall is not an option due to constraints within the environment (we recommend using a firewall), another option is to use ACLs within MySQL. This can be done by altering users to only be accessible via certain IPs or domains. This can be done using the following commands:

% charles here
\begin{lstlisting}[language=bash,frame=single,showstringspaces=false]
CREATE USER `user'@`localhost' IDENTIFIED BY `password';
CREATE USER `user'@`10.0.17.13' IDENTIFIED BY `password';
\end{lstlisting}

\noindent
\textbf{References}:

    \url{https://dev.mysql.com/doc/mysql-security-excerpt/8.0/en/general-security-issues.html}
    
    \url{https://dev.mysql.com/doc/mysql-security-excerpt/8.0/en/access-control.html}
    
    \url{https://www.digitalocean.com/community/tutorials/iptables-essentials-common-firewall-rules-and-commands}
\noindent

\newpage
\subsection{Moderate Risk}
%%%%%%%%%%%%%%%%%%%%%%%%%%%%%%%%%%%%%%%%%%%%%%%%%%%%%%%
%          Penetration Test Report Template           %
%                     cyber@cfreg                     %
%             https://hack.newhaven.edu/              %
%                                                     %
%                    Contributors:                    %
% TJ Balon - https://github.com/balon                 %
% Samuel Zurowski - https://github.com/samuelzurowski %
% Charles Barone - https://github.com/CharlesBarone   %
%%%%%%%%%%%%%%%%%%%%%%%%%%%%%%%%%%%%%%%%%%%%%%%%%%%%%%%
\subsubsection{Payment Transaction Enumeration}
\label{sec:payment_enumeration}

    \vulntext{4.5}
    
    % Description ================================================================================ 
    \color{black}{}
    \noindent
    \textbf{Description}:
    
    The Jawbreaker portal on the eggdicator host allows users to enter transaction IDs and returns transaction information including the amount, customer\_id, and status. All customer transactions can be enumerated without any authentication (see Figure \ref{fig:images/findings/moderate/example_moderate_finding.png}).
    
    \Figure[placement=h, width=\textwidth, frame, caption={Powershell Script to pull all Customer transactions.}]{images/findings/moderate/example_moderate_finding.png}
    
    % Impact ================================================================================ 
    \noindent
    \textbf{Potential Business Impact}:
    
    Potential attackers could extract all transactions regarding the revenue of \cptc. Although theses transactions only refer to the customer as a UUID, the value is unique to that customer and could be used to associate multiple transactions to specific purchasers. 
    
    
        \noindent
        \textbf{Affected Hosts}:
        
            Eggdicator (10.0.17.10)
    
    % Details ================================================================================ 
    \noindent
    \textbf{Exploitation Details}:
    
    The powershell script shown above can be used by navigating to \textit{https://10.0.17.10/payment/\$i} and replacing ``\$i'' with a numeric value.  This would return JSON data regarding a transaction if one exists with the given ID. Based on results obtained from the script, a total of 6469 transactions were present and extractable.
    
    % Remediation ================================================================================ 
    \noindent
    \textbf{Recommended Remediation}:
    
    Using a UUID instead of an id for each transaction would mitigate sequential enumeration and would greatly increase the time needed for a brute force attack. Additionally, rate limiting hosts to only a specific number of requests per minute (more information can be found in references) would further mitigate the attack. Moreover, implementing authentication on the API would limit enumeration to only authorized users.  Access tokens or basic http authentication are both options which would help reduce the risk introduced by this vulnerability.
    
    \noindent
    \textbf{References}:
    
    \url{https://www.nginx.com/blog/rate-limiting-nginx/}
    \url{https://docs.nginx.com/nginx/admin-guide/security-controls/configuring-http-basic-authentication/}
    
        %\url{https://nvd.nist.gov/vuln/detail/CVE-2019-9193#vulnCurrentDescriptionTitle}
    

\newpage
\subsection{Low Risk}
%%%%%%%%%%%%%%%%%%%%%%%%%%%%%%%%%%%%%%%%%%%%%%%%%%%%%%%
%          Penetration Test Report Template           %
%                     cyber@cfreg                     %
%             https://hack.newhaven.edu/              %
%                                                     %
%                    Contributors:                    %
% TJ Balon - https://github.com/tjbalon               %
% Samuel Zurowski - https://github.com/samuelzurowski %
% Charles Barone - https://github.com/CharlesBarone   %
%%%%%%%%%%%%%%%%%%%%%%%%%%%%%%%%%%%%%%%%%%%%%%%%%%%%%%%
\subsubsection{ScadaBR Reflected XSS (Username)}
    \label{sec:scadabr_xss_1_username}
    \noindent
    
    \vulntext{1.25}
    
    \noindent
    
    
    \color{black}{}
    \textbf{Description}:
    
    A cross-site scripting (XSS) vulnerability was found in the login portal of ScadaBR.  A malicious actor could supply malicious Javascript to redirect users and steal cookies.  
    
    \Figure[placement=h, width=\textwidth, frame, caption={XSS Vulnerability in username input field}]{images/findings/low/example_low_finding.png}
    
    \noindent
    \textbf{Potential Business Impact}:
    
    Because this vulnerability uses reflected XSS instead of stored XSS, phishing would most likely be required for a successful exploit.  However, an employee of \cptc falling for a phishing campaign could lead to drive-by downloads and cookie theft, resulting in the potential compromise of machines on the network.
    
    \noindent
    \textbf{Affected Host}:
    
    Crunch (10.0.17.50)
    
    Crunchserial (10.0.17.51)
    
    \noindent
    \textbf{Exploitation Details}:
    
    Results can be reproduced by navigating to \textit{http://10.0.17.50:9090/ScadaBR/} (without having been authenticated beforehand) and typing the following in the username field:
    
\begin{lstlisting}[language=bash,frame=single,showstringspaces=false]
admin"><script>alert("XSS")</script>
\end{lstlisting}

    This will result in a reflected cross-site vulnerability displaying an alert on the page with the text ``XSS''.  While the alert text is only an example, a threat actor could include arbitrary JavaScript in the payload.
    
    \noindent
    \textbf{Recommended Remediation}:
    
    Validate \& sanitize all form fields to prevent XSS attacks. Use a Web application firewall (WAF) to block the execution of malicious scripts. Additionally, convert all alphanumeric characters to HTML character entities before displaying user input. To ensure that cookies cannot be stolen, it is recommended to include in the headers ``Secure'' and ``HttpOnly'' so that cookies are not accessible to unintended parties and are sent over HTTPS. At the moment, the web traffic on the ScadaBR server is not encrypted. The Secure header can only be implemented once TLS is implemented and enabled on the host. 
    
    
    \noindent
    \textbf{References}:
    
    \url{https://developer.mozilla.org/en-US/docs/Web/HTTP/Cookies}
    
    \emph{CEHv11 Ethical Hacking and Countermeasures - Volume 2}
    

\newpage
\subsection{Informational}
%%%%%%%%%%%%%%%%%%%%%%%%%%%%%%%%%%%%%%%%%%%%%%%%%%%%%%%
%          Penetration Test Report Template           %
%                     cyber@cfreg                     %
%             https://hack.newhaven.edu/              %
%                                                     %
%                    Contributors:                    %
% TJ Balon - https://github.com/balon                 %
% Samuel Zurowski - https://github.com/samuelzurowski %
% Charles Barone - https://github.com/CharlesBarone   %
%%%%%%%%%%%%%%%%%%%%%%%%%%%%%%%%%%%%%%%%%%%%%%%%%%%%%%%
\subsubsection{Insufficient Firewalls}
\label{sec:insufficient-firewalls}
\noindent


\color{black}{}
\textbf{Description}:

During the engagement, many of the services and hosts interacted displayed insufficient Firewall Rules. 

\noindent
\textbf{Potential Business Impact}:
This could save potential incidents of breach as by enabling firewalls with strong rules could thwart attackers from gaining unauthorized access to systems that potentially contain security vulnerabilities such as remote code execution.

\noindent
\textbf{Recommended Remediation}:
On hosts setup up proper firewalls using iptables, ufw, or other software. Examples of these can be found in Section \ref{sec:mysql_no_auth}. PCI DSS requires firewall zone-based controls between trusted and untrusted zones. Some best practices for firewalls are to:

\begin{itemize}
    \item Block traffic by default
    \item Set Explicit Firewall Rules First 
    \item Establish firewall configuration change plan 
    \item Optimize firewall rules
    \item Update Firewall Software Regularly
\end{itemize}

Additionally, it may also be beneficial for \cptc\ to implement intrusion-detection and/or intrusion-prevention systems on the network to help with detecting and preventing future exploitation of the network.

\noindent
\textbf{References}:

\url{https://backbox.com/7-firewall-best-practices-for-securing-your-network/}
\url{https://www.checkpoint.com/cyber-hub/network-security/what-is-firewall/8-firewall-best-practices-for-securing-the-network/}


